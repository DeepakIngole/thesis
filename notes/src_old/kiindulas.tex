\section*{Feladat}

A félév során a DirectDrive áttétel nélküli szervóhajtás rendszerintegrációjával foglalkoztam. Előző félévben a szervóhajtás szabályzó köreit hangoltam, melyhez a gyártó által biztosított szoftver állt rendelkezésre, az Elmo Application Studio. Ezzel lehetett beállítani a szabályzást és kézi módban tesztelni az alapjelkövetést. Ez elkészült, a nyári gyakorlat során így megvizsgáltam, hogy tudja egy beágyazott rendszer irányítani a hajtást.  

\begin{figure}[H]
	\centering
	%\begin{subfigure}[t]{0.8\textwidth}
	\centering
	\includegraphics[width=12cm]{figures/hw/dd}
	\caption{A DirectDrive szervóhajtás}
	\label{fig:dd}
	%	\end{subfigure}
\end{figure}

%A szervóhajtás egy kísérleti repülőgépen fog helyet kapni, ahol a fedélzeti beágyazott számítógép fogja vezérelni. A szoftver funkcióit kell átültetni és az igényeinkre szabni, hogy ezek a automatizáltan végrehajthatók legyenek. A fedélzeti számítógép feladata a hajtás:

A hajtást irányító eszköz feladatai:
\begin{itemize}[noitemsep,topsep=-8pt,parsep=0pt,partopsep=-6pt]
	\item a motor engedélyezése, a mozgás paraméterezése
	\item referenciajel kiadása
	\item pozíció és egyéb adatok visszamérése
\end{itemize}

A szabályzó hangolása és az alapjelkövetés tesztelése során az alábbi ábra szerinti eszközöket használtam. A szoftverben egy gombbal lehet a motort engedélyezni, a mozgás paraméterei részletesen beállíthatók, az encoder által mért pozíciót annak inkremenseiben leolvashatjuk. 

\begin{figure}[H]
	\centering
	\begin{tikzpicture}[->,>=stealth']
\centering
% Position of FCC 
% Use previously defined 'state' as layout (see above)
% use tabular for content to get columns/rows
% parbox to limit width of the listing
\node[state,
yshift=0cm,
text width=3cm] (PC) 
{
	\begin{tabular}{l} 	% content
		\textbf{Számítógép}\\
		\parbox{2.8cm}{	Elmo Application Studio}
	\end{tabular}
};


% State: TWI with different content
\node[state,    	% layout (defined above)
text width=3cm, 	% max text width
%yshift=2cm, 		% move 2cm in y
right of=PC, 	% Position is to the right of QUERY
node distance=5cm, 	% distance to FCC
anchor=center] (TWI) 	% posistion relative to the center of the 'box'
{%
	\begin{tabular}{l} 	% content
	\textbf{Twitter}\\
	\parbox{2.8cm}{motorvezérlő}
	\end{tabular}
};

% STATE ENC
\node[state,
right of=TWI,
node distance=3.5cm,
yshift=-1.75cm,
anchor=center,
text width=3cm] (ENC) 
{%
	\begin{tabular}{l}
	\textbf{Encoder}\\
	%\parbox{2.8cm}{Dekrementiere Slotzähler}
	\end{tabular}
};

% STATE BLDC
\node[state,
right of=TWI,
yshift=1.75cm,
node distance=3.5cm,
anchor=center] (BLDC) 
{
	\begin{tabular}{l}
	\textbf{BLDC motor}\\
	%\parbox{4cm}{Mit nächstem \mbox{\textbf{QueryRep}} als "`inventoried"' markieren.}
	\end{tabular}
};

% draw the paths and and print some Text below/above the graph
\path
(PC)		edge[<->]  	node[anchor=north,above]{Serial} 		(TWI)
(TWI)  	edge[dashed]                               				(BLDC)
(BLDC) 	edge[dashed]                            				(ENC)
(ENC) 		edge												(TWI);

\end{tikzpicture}
	\label{fig:control}
	\caption{Szervóhajtás tesztje a gyártói szoftverrel}
\end{figure}
%Ezeket eddig a szoftverben grafikus felületről lehetett kezelni illetve leolvasni. 
%Azokat az információkat is fel kell dolgozni, amik a szoftverből leolvashatók voltak, mivel a referenciajelet egy magasabb szintű szabályzás számítja.
%Így meg kellett valósítani a szabályzó tesztelése során szoftverből, grafikus felületen használt funkciókat.


\begin{figure}[H]
%	\begin{subfigure}[t]{0.8\textwidth}
	\centering
	\includegraphics[width=13cm]{figures/measure/elmo}
	\caption{Mozgás végrehajtása grafikus szoftverből}
	\label{fig:elmo}
%	\end{subfigure}
\end{figure}

A hajtás egy kísérleti repülőgépen fog helyet kapni, ahol szöghelyzet referenciajelet kap majd a fedélzeti számítógéptől. A nyári gyakorlatom során a fenti szoftver helyett a motorvezérlővel CANopen protokoll szerinti üzenetekkel kommunikáltam.
%\vspace{12pt}

% {\color{red}\hrule}