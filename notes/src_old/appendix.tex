	\section*{Appendix}\label{appendix}
	
	\subsubsection*{The default position of the control switches on the JETI
	transmitter}\label{the-default-position-of-the-control-switches-on-the-jeti-transmitter}
	
	\begin{itemize}
	\itemsep1pt\parskip0pt\parsep0pt
	\item
	  Sticks are centered (labelings: P1, P2, P3, P4), side arms are in
	  middle (center) position (labels: P5, P6) - this four lever is
	  relevant
	\item
	  Turn the switch labeled \emph{SD} to off state: this turns autopilot
	  off. The states are indicated via labeling on the transmitter
	\end{itemize}
	
	\subsubsection*{Manual Flight Mode}\label{manual-flight-mode}
	
	On the RX-MUX unit, the red LED has three states:
	
	\begin{itemize}
	\itemsep1pt\parskip0pt\parsep0pt
	\item
	  Waiting for RC signal - Periodic blinking with 1Hz
	\item
	  Autopilot mode on/Manual off - LED off, static
	\item
	  Autopilot mode off/Manual on - LED on, static
	\end{itemize}
	
	On the Sindy FCC stack, one RX-MUX board is present but one board
	consists of two separated circuits controling the servos on the left and
	right side independently. (left side marked with \emph{\texttt{1}} on
	PCB / fuselage cover and \texttt{I.} at BEC, right is the
	\emph{\texttt{2}}/\texttt{II.})
	
	\subsubsection*{Connection to the Raspberry
	Pi}\label{connection-to-the-raspberry-pi}
	
	\begin{itemize}
	\itemsep1pt\parskip0pt\parsep0pt
	\item
	  Connect to the Raspberry Pi via Ethernet cable, or directly to the
	  local WiFi of the Raspberry Pi
	
	  \begin{itemize}
	  \itemsep1pt\parskip0pt\parsep0pt
	  \item
	    Find the local WiFi network of the FCC - That should be labeled as
	    \textbf{FCC\_Sindy\_AP} or \textbf{FCC\_HIL\_AP}. Connect to that
	    network.
	  \item
	    Find the IP address of the FCC - using Linux terminal commands
	
	    \begin{itemize}
	    \itemsep1pt\parskip0pt\parsep0pt
	    \item
	      Figure out your current IP address with the following command `ip
	      addr show'. Your IP address will be after the inet tag in the data
	      stream. For example, on the network of \textbf{FCC\_Sindy\_AP},
	      your IP should be something like: 10.0.0.\#.
	    \item
	      To find the FCC itself, you have to find every other device
	      connected to the network you are currently on. In our case, the
	      only other device is the FCC itself. Use the `nmap -sP
	      10.0.0.\#/24' command to list all of the devices on your network.
	      The other IP on the network will be the FCC. Use this IP address
	      in the process of creating the ssh tunel, instead of the
	      `raspberrypi.local' hostname. On the \textbf{FCC\_Sindy\_AP} the
	      IP of the FCC should be something like: 10.0.0.1
	    \end{itemize}
	  \end{itemize}
	\item
	  Create an ssh tunnel to the Raspberry Pi via the local network, use
	  the following command \texttt{ssh pi@raspberrypi.local} use
	  \textbf{hallgato} or \textbf{raspberry} as the password
	\item
	  Start the FCC software with the \texttt{sudo ./SW\_PI} command
	\item
	  You can unplug the cable/disconnect from WiFi, but \textbf{do not
	  close} the SSH tunnel!
	\end{itemize}
	
	\subsubsection*{Checking the Center of
	Gravity}\label{checking-the-center-of-gravity}
	
	The center of gravity of the aircraft is an important thing for the
	stability of the aircraft itself. If the aircraft is balanced in the
	right way, then it is possible for it to glide smoothly in a
	disturbance-free area if the control surfaces in the neutral position.
	To check the center of gravity of the aircraft follow these steps - you
	will need one person at each wing to follow these instructions:
	
	\begin{itemize}
	\itemsep1pt\parskip0pt\parsep0pt
	\item
	  Find the front spar in the wing - there is a blue marker on the lower
	  skin alongside the front spar
	\item
	  Put your hand unto the line of the front spar under the wing.
	\item
	  Raise it up from the ground
	\item
	  If the air craft is in balance - does not pitch, or stays in the same
	  orientation as you set it - then the center of gravity is
	  approximately in the right position
	\end{itemize}
	
	\begin{center}\rule{3in}{0.4pt}\end{center}
	


