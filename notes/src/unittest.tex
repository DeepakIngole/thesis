%\chapter{Modellek tesztje}
%
%\subsection{Radiátor unit test}
%
%\subsubsection{Állandósult állapot numerikus modellje}
%
%Annak ellenőrzése, hogy a \ref{holeadas4} egyenlet jó-e. Azaz elfogadható-e ez a közelítés állandósult állapotban, illetve a tranziens alatt mennyire feasible. 
%
%Az egyenletben a mintavételi idő egy szorzóként jelenik meg, 
%
%Az egyenlet wattban adja a kimenetét.
%A teszt egy formája lehet, ha a gyári adatokat (fűtési teljesítmény) összevetem az általam számoltakkal.
%
%\subsubsection{Tranziens Simscape modellje}
%A bejövő hő függvényében a hőleadás tranziensei. A bejövő hőt a képlet numerikusan számítja. A tranzienst viszont Simscape-ben szimulálom. Ez folytonos rendszert feltételez.
%
%\subsubsection{Szabályzás célja}
%
%Állandósult állapotban olyan bemenő hőáramot elérni, ami épp fedezi a veszteségeket.
%
%\subsection{Padlófűtés unit test}
