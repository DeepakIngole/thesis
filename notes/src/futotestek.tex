\section{Fűtési rendszerek modellje}


\subsection{Általános összefüggések, ill. a radiátor modelljének felírása}

Mivel a Matlab szimulációban a legbefúvásos fűtés modelljének teljesítmény kimenete van, fel akartam állítani egy olyan modellt, ami beillesztehető az eredeti légbefúvó rendszer helyére. A ház hőveszteségeit a Matlab számolja\footnote{Pontosításra szorul ez a modell is, mert valószínűleg csak a konvektív hővezetéssel számol (a sugárzásival pedig nem). A légbefúvás a ház levegőjét melegíti. Ám a modellben a ház hőtároló tömege nem jelenik meg, csak egy hőellenállás a veszteségek modellezéséhez.}, ebből pedig adódik a szoba levegőjének hőmérséklete. A rendszer szabályozását így visszavezettem a leadott teljesítmény szabályzására. A levezetett egyenletnek köszönhetően egy teljesítményigényhez meg tudom majd mondani hogy mennyire kell a szabályzószelepeket kinyitni.

%Angol nyelvű szakirodalomból pl. Gouda2000 alapján számolva irreális teljesítményértékeket kaptam (150kW), tovább keresve magyar nyelvű irodalmat is áttekintettem.

Az \textit{Épületgépészet a gyakorlatban}\footnote{Könyvtári könyv, Verlag. 5.11.6, 2. o.} c. könyvben szó esik fűtési rendszerek méretezéséről. Itt adatként szerepel egy épületre a szobák hőigénye\footnote{Pontosan nem tudom még, hogyan definiálják a hőigényt: mekkora kültéri hőmérsékletet vesznek pl. figyelembe, illetve hogy radiátor méretezésénél ezt nyilván felül kell becsülni.} és névleges hőmérséklete. Ehhez választanak megfelelő méretű radiátort, hogy azokban a kiszámolt sebességgel vizet keringetve a hőleadás elég legyen az adott helyiségbe.
{\scriptsize(Ehhez figyelembe kell venni minden radiátorra a keringő víz hőmérsékletét is, különösen ha azok sorba vannak kötve és a hőmérsékletesések is jelentősek.)}
% Adottnak véve az előremenő és visszatérő hőmérsékletet az összes hőigényből számolható a víz kívánt áramlási sebessége. Ezután meghatározzák a radiátorok méretét, hogy azoknak a hőleadása megfeleljen az előírtaknak.

%A fenti példák segítenek a modellalkotásban is, felírható a radiátorok teljesítménye változó vízhőmérséklet és víz tömegáram esetén is. Természetesen a modell egyik bemenete, ez esetben a tömegáram a szabályzott mennyiség. Felteszem, hogy ezt folytonosan tudjuk szabályozni egy szelep segítségével (vagy ha ez nem életszerű, akkor kétállású szeleppel, de nagyobb frekvenciával, mint ahogy egy kazánt tudnánk ki/be kapcsolni).

Hasonlóan méretezési feladatot mutat be a \cite[4.2.7.3]{Herz} is. Ezek alapján vezettem le a leadott hő mennyiségét állandósult állapotra. Természetesen a felmelegedés és lehűlés idejét is figyelembe kell majd venni, de ezzel érthető módon a méretezésnél sem számolnak.%További egyszerűsítésként elhanyagoltam a hőleadási tényező hőmérsékletfüggését is.
%Itt a hőveszteség adott. Esetünkben ezt a házra a Matlab számolja és jól méretezett rendszert tételezünk fel. Csupán azért kell a hőleadást jól felírni, hogy a felfutás, hőkapacitás, stb. során átadott energiát is belekalkuláljuk.

%Persze ilyenkor egyedi esetekből indulok ki, de remélhetőleg ez paraméterezhetően elvezet az általános, többféle házra alkalmazható megoldáshoz.

\subsubsection{Hőleadás}
A fűtőtestek hőleadását befolyásolja a fűtőtestek közepes hőmérsékletkülönbsége (ld. a \ref{termeszeteshk_359}. egyenletet), a felülete és a hőleadási tényezője.
%(A 86. oldalon $\Delta t_k$, a 358.-on $\Delta t_m$ jelöléssel találkozunk. A \cite[359.~o.]{Herz} ismét változik ugyanannak a jelölése. (\ref{termeszeteshk_359}) Ezutóbbi angol jelölés szimpatikusabb.)
%
Ezek közötti kapcsolatot adja az \ref{holeadas}. egyenlet (\cite[358.~o.]{Herz}-ből): 
\begin{equation} \label{holeadas}
\dot Q_{le} = k_e ~ A_e ~ \Delta t_m
\end{equation}
%
%
ahol
\begin{itemize}[itemsep=6pt,topsep=0pt,parsep=0pt,partopsep=0pt]
\item[] $\dot{Q}_{le}$ [\SI{}{\watt}] a leadott hő
\item[] $k_e$ [\si[per-mode = fraction]{\watt\per\meter\squared\per\kelvin}] hőleadási tényező - ezt hőmérsékletfüggetlennek tekintem.
\item[] $A_e$ [\si{\metre\squared}] a radiátor felülete
\item[] $\Delta t_m$ [\SI{}{\kelvin}] a közepes hőmérsékletkülönbség:
\end{itemize}
%
\begin{equation} \label{termeszeteshk_359}
\Delta t_m = \frac{t_s+t_r}{2} -t_{i}
\end{equation}
ahol
%
\begin{itemize}[itemsep=6pt,topsep=0pt,parsep=0pt,partopsep=0pt]
	\item[] $t_s$ a radiátorba befolyó, $t_r$ az onnan kifolyó víz hőmérséklete \si{\degreeCelsius}-ban
	\item[] $t_i$ a szoba hőmérséklete
\end{itemize}
%
A hőátadási tényező is hőmérsékletfüggő, de ezzel egyelőre nem foglalkozom, állandónak tekintem.
%
%\begin{equation} \label{k_e}
%k_e = \frac{\dot{Q}}{A~ \Delta t_m}
%\end{equation}
%
%A hőteljesítmény hőmérsékletfüggő (361.~o.). Az $x^{1.33}$ az egyenletekben $x~ x^{1/3}$, ebből pedig $ x ~ \sqrt[3]{x}$ formában jelenik meg.
%
%
%Nominálisan $\Delta t_m$ = \SI{60}{\celsius}-ra adott érték a közepes hőmérsékletkülönbség függvényében változik:

\subsubsection{Hőfelvétel}
A vízből felvett hő felírható:

\begin{equation} \label{hofelvetel}
\dot Q_{fel} = c ~ \dot{m} ~ \Delta t
\end{equation}

ahol

\begin{itemize}[itemsep=6pt,topsep=0pt,parsep=0pt,partopsep=0pt]
	\item[] $\dot{Q}_{fel}$ [\SI{}{\watt}] a vízből felvett hő, ami annak lehűléséből adódik
	\item[] $c$ [\si[per-mode = fraction]{\joule\per\kg\per\kelvin}] a víz fajhője
	\item[] $\dot{m}$ [\si[per-mode = fraction]{\kg\per\second}] a víz tömegárama
	\item[] $\Delta t = t_s-t_r$ [\SI{}{\kelvin}] a víz lehűlésének mértéke
\end{itemize}

\subsubsection{Hőkapacitás}

Katalógusból radiátorok tömege és a bennük lévő víz térfogata leolvasható. A hőkapac számítása:

\subsubsection{Energiamérleg állandósult állapotban}
\textbf{Állandósult állapot} esetén a leadott hő egyenlő a felvettel, mivel akkor nem történik hőfelhalmozás, hőtárolás.
Azaz ekkor a radiátor hőkapacitását nem kell figyelembe vennem.

Beírva a (\ref{termeszeteshk_359})-ba (\ref{holeadas})-t:
\begin{equation} \label{holeadas2}
\begin{aligned}
\dot Q_{le} = k_e ~ A_e ~ \left( \frac{t_s+t_r}{2}-t_i\right) = k_e ~ A_e ~ \left( \frac{t_s+(t_s-\Delta t)}{2}-t_i\right)
\end{aligned}
\end{equation}

Ahol felhasználtuk azt is, hogy $t_r = t_s-\Delta t$, majd $\Delta t$ helyére beírhatjuk a (\ref{hofelvetel})  átrendezett alakját:
\begin{equation} \label{hofelvetel2}
~~\Delta t = \frac{\dot Q_{fel}}{c ~ \dot{m}}
\end{equation}

Beírva (\ref{holeadas2})-ba (\ref{hofelvetel2})-t:
\begin{equation} \label{holeadas3}
\begin{aligned}
\dot Q_{le} ~=~ & k_e ~ A_e ~ \left( t_s-t_i-\frac{\dot Q_{fel}}{c ~ \dot{m}}\right)  \\[18pt]
\dot Q_{le} + \frac{k_e ~ A_e ~ \dot Q_{fel}}{2 ~ c ~ \dot{m}} ~ = ~ & k_e ~ A_e ~(t_s-t_i) \\[24pt]
2 ~ c ~ \dot{m} ~ \dot Q_{le} + k_e ~ A_e ~ \dot Q_{fel} ~ = ~ &  k_e ~ A_e ~ 2~ c~ \dot{m} ~(t_s-t_i)
\end{aligned}
\end{equation}

\textbf{Csak abban az esetben, ha} $\dot Q_{le}=\dot Q_{fel}$:

%(meggondolandó hogy a hőkapacitások szerepe hogy alakul...)


\begin{equation} \label{holeadas4}
\begin{aligned}
~~~~~~\dot Q (2 ~ c ~ \dot{m} + k_e ~ A_e) & ~=~ 2 ~ k_e ~ A_e ~ c~ \dot{m} ~(t_s-t_i) \\[18pt]
~~~~~~\dot Q &~=~ \frac{2~c~\dot{m}~k_e~A_e}{2 ~c ~ \dot{m} + k_e ~ A_e}~(t_s-t_i)
\end{aligned}
\end{equation}

A fenti képletet kiegészítve kezelhető lenne a hőmérsékletfüggő hőleadási tényező.

Mivel a hőleadást, hőtárolást Simscape-ben valósítottam meg, a radiátorba bemenő hőt kell csak kiszámítani. Erre meg kell vizsgálni, hogy az állandósult állapotbeli képlet helyes-e.

\begin{formal}
	\textbf{Megjegyzés:} A radiátorba bekerülő teljesítményt a $t_s-t_r$ szabja meg (\ref{hofelvetel}. egyenlet), viszont itt $t_r$-t kiejtettem az egyenletekből. Viszont az irodalom\footnote{REHVA alacsony hőm. fűtés és magas hőm. hűtés by Bjarne Olesen et. al.} szerint a $\Delta t= t_s-t_r$-re szabályzozással megtakarítás érhető el. Meg kell vizsgálni, reális-e mindkét paraméter mérése, radiátorok esetén, vagy csak padlófűtésnél.
\end{formal}

\subsubsection{Javítás a radiátormodellen}

A közepes vízhőmérséklet, a közepes felületi hőmérséklet is jöhet kimeneten ahhoz, hogy a steady-state model számolhassa a bemenő hőmérsékletet.


\subsubsection{Dinamikus modell}

A felmelegedéskor és lehűléskor a pontos hőleadást akkor tudjuk modellezni, ha ismerjük a radiátor hőkapacitását. Ehhez tudnunk kell, hogy a radiátorban mennyi víz van, illetve hogy a radiátortest milyen nehéz.

Radiátor katalógusokból\footnote{Purmo Ventil Compact - purmo.com} azt találtam, hogy az egyes radiátor típusokra ezek a paraméterek milyen értékűek.

Ismert a radiátor hossza, magassága, konstrukciója. Ezalapján a
tömege, illetve az acél hőkapacitása alapján a radiátortest hőkapacitása - simscape termikus hőtároló elem blokként
víztérfogata, a víz fajhője még egy hőtároló elem.

Ezen hőtároló elemek feltöltődése szimulálva adja a dinamikus viselkedést.

%A modell kimenetén külön szerepelhet a sugárzás és a konvekció.

\subsection{Padlófűtés modellje}

Aljzat, aljzatbeton: slab
facade: frontal - homlokzat


\subsubsection{Sugárzó és konvektív teljesítmény szétválasztása}

Fun facts:
~
\begin{itemize}[itemsep=6pt,topsep=0pt,parsep=0pt,partopsep=0pt]
	\item A falakra az $\alpha$ = 10 \si[per-mode = fraction]{\watt\per\meter\squared\per\kelvin} érték a sugárzó és konvektív hőleadást is tartalmazza. A konvektív hőleadás függ a felületi áramlási sebességtől: falsaroknál ez az érték alacsonyabb, kb. a fele.
	\item A sugárzó hő a Stefan-Boltzmann törvény alapján függ az emisszivitástól. (Annak a mértéke, hogy a test a feketetesthez képest mennyi hőt bocsát ki). A hőmennyiség a hőmérséklet negyedik hatványával arányos. A \textbf{sugárzott hő meghatározásához} még meg kell keresni és be kell írni a Simscape blokkba a megfelelő együtthatókat. Valami általános összefüggést kell találni, hogy a radiátor milyen arányban melegíti a külső falat, ahol van, ill. az ablakra milyen hatással van: még nem kezelem le ezeket az aszimmetriákat, hanem minden hőmérsékleteloszlást homogénnek veszek. A Stefan-Boltzmann törvény direkt alkalmazása helyett a szabványokban és irodalomban található közelítésekkel élek.
	\item A $q_r$ [\si[per-mode = fraction]{\watt\per\meter\squared}] \textit{radiant heat flux density} a T. Cholewa\footnote{On the heat transfer coefficients between heated/cooled radiant floor and room. \\ DOI: http://dx.doi.org/10.1016/j.enbuild.2013.07.065} (5.) egyenlet alapján számítható de az a geometriától is angyban függ. Helyette Kilkis1994 (4) és (6) javasolt, illetve a \cite{CHOLEWA2013599}-ból is lehet mért értékekkel számolni / a szabványok ajánlását használni.
	\item A hőhidak a hőveszteségek meglepően nagy részéért felelősek, jelentős hibát követünk el, ha ezekkel nem számolunk. Meg kell keresni az energ. tanúsítványokban hogy hol tüntetik fel ezek mértékét.
	
	
\end{itemize} 

Fűtött padló, falak, mennyezet esetén jelentős szerepe van a sugárzó hőleadásnak.

\begin{itemize}[itemsep=0pt,topsep=0pt,parsep=0pt,partopsep=0pt]
	\item A.Laouadi / Building and Environment 39 (2004) 421 – 431 - p424, eq. 10-11: radiant heat transfer model
	\item TEMPERATURE CONTROL STRATEGIES FOR RADIANT FLOOR EIEATING SYSTEMS, Zhi Long Zhang: 40.o.  
	\item \cite{CHOLEWA2013599} T. Cholewa et al. / Energy and Buildings 66 (2013) 599–606 - Table 5: coefficient
	\item Kilkis1994 A simplified model for radiant heating and cooling panels: itt van képlet sugárzóra
	\item Kiegészítés: \cite[349.~o.]{Herz}
\end{itemize}  

A sugárzó hőleadási tényező bevezetésével viszont linearizálhatjuk a hőleadást, a hőleadás így egyszerűen lineárisan függ majd a hőmérséklet-különbségtől.

\begin{equation} \label{equation-radi-and-convective-htotal}
\dot Q_{r} = h_r ~ A_e ~ \left(t_{surf}-t_{AUST}\right)
\end{equation}

ahol
\begin{itemize}[itemsep=3pt,topsep=0pt,parsep=0pt,partopsep=0pt]
	\item[] $\dot{Q}_{r}$ [\SI{}{\watt}] a leadott sugárzó hő
	\item[] $h_r$ [\si[per-mode = fraction]{\watt\per\meter\squared\per\kelvin}] sugárzó hőleadási tényező
	\item[] $A_e$ [\si{\metre\squared}] a padló felülete
	\item[] $t_{surf}$ [\SI{}{\kelvin}] padló hőmérséklete
	\item[] $t_{AUST}$ [\SI{}{\kelvin}] fűtetlen felületek átlagos hőmérséklete - a fal hőmérsékletének veszem a Simscapeben
\end{itemize}

\subsubsection{Problémák, eltérések a radiátoros fűtéshez képest}

%Hasonló képlettel számolok, mint a radiátornál.
Tegyük fel ho  

\pagebreak