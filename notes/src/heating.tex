\section{Fűtőtestek modellezése}

A MATLAB egyik demójában egy ház fűtési modelljét valósították meg.
Ebben a fűtőtest kimenete teljesítmény dimenziójú. 
A ház veszteségeit a méretei és az ablakai alapján kiszámítja.

A modellezendő objektum a fűtési rendszer, itt kell számba venni hogy egy jól méretezett rendszernek mennyi a felfutási illetve a beállási ideje. Ezt számolni a kazán, a fűtővíz illetve a fűtőtest teljesítményeiből, illetve kapacitásaiból lehetne.

A fűtőtestek hőátadását számos tényező befolyásolja, ezekre az előzóekben egyenletet is felírtam. Az egyenletet Simulinkben valósítottam meg, a fűtési rendszer így beilleszthető a Matlab példájába.
%, paraméterei pl.
%\begin{itemize}[noitemsep,topsep=0pt,parsep=0pt,partopsep=0pt]
%	\item a bejövő és kimenő víz hőmérséklet-különbsége (radiátoros / falfűtés / padlófűtés / mennyezetfűtés)
%	\item azok területe, orientációja, azokon a légáramlás sebessége, típusa
%	\item 
%\end{itemize}

%Először célszerű lenne állandósult állapotot figyelembe venni és úgy számolni a teljesítményeket.
%
%Kezdetben fel kellene írni átviteli függvényeket az alábbiakra:



%\begin{itemize}[noitemsep,topsep=0pt,parsep=0pt,partopsep=0pt]
%	\item fűtőtest hőleadása a vízhőmérséklet függvényében
%	\item vízhőmérséklet (felfűtés) időfüggvénye
%	\item forró fűtőtest lehűlése a keringetés kikapcsolásával
%	\item 
%	\item 
%\end{itemize}

