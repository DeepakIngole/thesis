%\section{Fűtési rendszer}
%
%
%\begin{itemize}[noitemsep,topsep=0pt,parsep=0pt,partopsep=0pt]
%	\item szabályzó
%	\item kazán: bemenete on/off (plusz egy egytárolós taga az elejére)
%	\item szelep: 0..1 folytonosnak tekintve. Hihi-haha, kvantálási hiba megjelenhet. (Ha csak 10 lépcsős a szelep???)
%	\item megfontolandó bemenet a vízhőmérséklet, hiszen az szokott változni a külső hőmérséklet függvényében.
%	\item a fűtőtesteknek szintén van hőtároló képességük
%	
%\end{itemize}




%, paraméterei pl.
%\begin{itemize}[noitemsep,topsep=0pt,parsep=0pt,partopsep=0pt]
%	\item a bejövő és kimenő víz hőmérséklet-különbsége (radiátoros / falfűtés / padlófűtés / mennyezetfűtés)
%	\item azok területe, orientációja, azokon a légáramlás sebessége, típusa
%	\item 
%\end{itemize}

%Először célszerű lenne állandósult állapotot figyelembe venni és úgy számolni a teljesítményeket.
%
%Kezdetben fel kellene írni átviteli függvényeket az alábbiakra:



%\begin{itemize}[noitemsep,topsep=0pt,parsep=0pt,partopsep=0pt]
%	\item fűtőtest hőleadása a vízhőmérséklet függvényében
%	\item vízhőmérséklet (felfűtés) időfüggvénye
%	\item forró fűtőtest lehűlése a keringetés kikapcsolásával
%	\item 
%	\item 
%\end{itemize}

%\pagebreak
