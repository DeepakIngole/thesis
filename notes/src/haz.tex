\chapter{Ház modellje}

% Kiindulás ssc_house_heating_system

A szabályzótervezéshez rendelkezésre kell, hogy álljon a szabályzott szakasz modellje. Ehhez egy könnyen módosítható, koncentrált paraméterű hálózatot vettem fel,
%. Felépítettem egy hálózatot\footnote{Fodor HáRe alapján a különbség rendszer és hálózat között: a hálózat a rendszer egy lehetséges megvalósítása. A Simscape blokkok összekapcsolva hálózatot alkotnak. Az Ident toolbox-szal egy átviteli függvényt identifikálok, az így kapott rendszer viselkedése valamilyen pontossággal megfelel a hálózaténak, amiből kiindultunk.},
ahol minden elemhez lehet fizikai tartalmat rendelni. Ahhoz, hogy ehhez szabályzót lehessen tervezni, identifikáltam azt az ugrásválaszával. %a Simscape használatával

A modell paraméterei ideális esetben az épület energetikai tanúsítványából kiolvashatók. Így a Matlab-ba ezeket az adatokat felvéve, és a tervezési lépéseket ott elvégezve a szabályzót rögtön használatba vehetjük, nincs szükség hosszas kalibrációs időszakra beüzemelésnél. A modellbeli eltéréseket esetlegesen kompenzálni lehet később, mérési adatok felhasználásával.% (historikus adatok felhasználásával). %A modellben a bizonytalanságok hatása adaptív szabályozással kezelhető (lesz).

\section{A modellalkotás folyamata}
%
%White-box
%grey-box
%black-box

% zavarás, bemenet, nem irányított bemenet

A modellt többféle úton is megkaphatjuk, de mindegyiknek lesz előnye és hátránya is: fizikai összefüggések alapján felírhatunk egy olyan modellt, amiben a hatásmechanizmusok pontosan szerepelnek, de a paraméterek a ténylegestől eltérőek lehetnek. Ha ezek nem ismertek, fekete dobozként ("black-box") is kezelhetjük a rendszert, de az identifikációhoz nagyon sok mérésre van szükség, hogy a mérési hibákat, zavarásokat kiküszöbölhessük.

Én a fizikai modell felírását választottam, de a szabályzótervezéshez így is egy átviteli függvényt használok. Ehhez azonban mégis sokkal egyszerűbb eljutni, mint mérésekkel:
a Simulinkben megvalósított hálózatra az identifikáció sokkal egyszerűbb, mint valós rendszerre. A vizsgálójelek tetszőlegesen megválaszthatók, pl. a külső hőmérséklet hatása is pontosan meghatározható. A ház egy MISO rendszer, terepi méréseket használva csak hosszas mérésekből lehet szétválasztani a bemenetek (fűtés, külső hőmérséklet, napsütés) hatását a belső hőmérsékletre. Ha később páratartalom szabályzása is szóba kerül, még bonyolultabb a helyzet.


%\cite{SCHIRRER201686}

% --------------------------- kihúzva:
%A szakirodalomban pl. \cite{THIEBLEMONT2017485} és \cite{SCHIRRER201686} érinti ezt a kérdést:
%
%A szabályzótervezés során néhányan egyáltalán nem alkotnak modellt, csak a mért adatokat használják fel, ami eléggé időigényes: az identifikációhoz egy megfelelően nagy amplitúdójú vizsgálójelre van szükség. Viszont egy 10\si{\celsius}-kal felfűteni egy helyiséget hosszú ideig tart, ami alatt biztosan meg fog változni pl. a külső hőmérséklet. Ha a mérésekben a különböző inputok hatása a kimeneten nem különíthető el jó, az identifikáció nehéz lesz. Illetve külső hőmérséklet sem változik ugrásszerűen, a lassú időbeli változás \textit{nem jó vizsgálójel} identifikációhoz.
% ---------------------------

%Lényegében én is mért adatokat használok NEMMM, EZÉRT VAN A TANÚSÍTVÁNY!, tulajdonképpen, mivel a modellt olyan alakban kéne felírni, hogy a szabályzó azt futtatni tudja. (?)


% ---------------------------


%Vizsgálójel kiválasztása
%
%Modell struktúra kiválasztása - átviteli függvények pólusainak, zérusainak száma
%
%
%Viszont az ident toolbox tf identjénél kihasználtam azt, hogy a rendszer jellegét ismerem, azaz hogy hány pólusa és hány zérusa van a szakasznak  / felnyitott körnek. Így lett egy nagyon jól illeszkedő átviteli függvényem.
%
%Én összeraktam a fizikai modellt simulinkben (ez white-box) majd annak az ugrásválaszát mértem. Így nem egy állapotteres modell, hanem egy átviteli fv. "keletkezett".


Egyzónás hőmérsékletszabályzást veszek alapul, azaz egy referenciajelem és egy mért hőmérsékletem van, a modellben a szoba levegőjének hőmérsékletét mindenhol ugyanakkorának feltételezem. A szabályzás külső behatások ellenében történik, úgy mint alacsonyabb külső hőmérséklet, illetve a napsütés, szellőzés hatása. Nem foglalkozok viszont belső zavarással, pl. több szoba különböző típusú fűtésével, vagy a belső hőterheléssel, ami pl. emberek jelenlétéből fakad.

Természetesen lehetett volna nagyon sok állapotú állapoteres modellt is létrehozni, ám rengeteg nem mérhető belső változója lett volna, emiatt nem biztos hogy teljesen irányítható vagy megfigyelhető rendszert kaptam volna, így pedig a szabályzótervezés nem működik.
 
 \textbf{ÁBRA WHITE-BOX MODELRŐL, ÉS BLACKBOX IDENTIFIKÁCIÓRÓL. ILLETVE WHITEBOX IDENT.}


\section{Épületfizikai alapösszefüggések}
 
 A fizikai modell felírásához szükség van néhány alapösszefüggésre.
 
 \vspace{18pt}
 
\subsection*{Hővezetés, hőáramlás, hősugárzás}

\begin{table}[ht]
	\footnotesize
	\centering
	\caption{Hőközlés fajtái}
	%\renewcommand{\arraystretch}{2} % to increase cell height
	%\taburulecolor{gray}
	%\begin{tabular}{|p{0.8cm}|p{1cm}|p{1cm}|p{1cm}|p{1cm}|p{1cm}|p{1cm}|p{1cm}|}
	%
	\newcolumntype{C}[1]{>{\centering\arraybackslash}p{#1}}
\newcolumntype{R}[1]{>{\raggedleft\arraybackslash}p{#1}}


\begin{tabu}{p{1.5cm}C{1.6cm}C{7cm}C{4cm}}
	%{p{1.5cm}|C{0.8cm}|C{0.8cm}|C{0.8cm}|C{0.8cm}|C{0.8cm}|C{0.8cm}|C{0.8cm}|C{0.8cm}|}
	%\multicolumn{1}{l}{}&\multicolumn{8}{l}{SDO header (első adatbyte) - master kérése}
	%\\ 		\cline{2-9}\cline{2-9}
	\hline
	\\
	& együtthatója &  a hőátadás szereplői & példa
	\\
	konvektív &  $\lambda$& áramló közeg -- szilárd anyag felülete & levegő vagy víz áramlása
	\\
	konduktív &  $h_c$&  az anyag molekulái között & az anyag belsejében
	\\
	radiatív &  $h_r$& tárgyak között, felszínükkel arányosan & hősugárzás 
	\\
%	& méret & $h_t$, átlag    & hőtároló tömeg & hőkapac
%	\\
%	& méret & $h_t$, átlag    & hőtároló tömeg & hőkapac
%	\\
%	& méret & $h_t$, átlag    & hőtároló tömeg & hőkapac
%	\\
%	& méret & $h_t$, átlag    & hőtároló tömeg & hőkapac
%	\\
%	& méret & $h_t$, átlag    & hőtároló tömeg & hőkapac
%	\\ %\hline
%	külső fal & 4.5 \si{\metre\squared} & 2 \si[per-mode=fraction]{\watt\per\metre\squared\per\kelvin} & 900kg & 756 \si[per-mode=fraction]{\kilo\joule\per\kelvin}
%	\\ %\hline
%	ablak & 4 \si{\metre\squared} & 4 \si[per-mode=fraction]{\watt\per\metre\squared\per\kelvin} & 0 & 0
%	\\ %\hline
%	belső válaszfalak & 50 \si{\metre\squared} & 7 \si[per-mode=fraction]{\watt\per\metre\squared\per\kelvin} & 5000kg & 4.2 \si[per-mode=fraction]{\mega\joule\per\kelvin}	
%	\\ %\hline
%	padló & 16 \si{\metre\squared} & 11 \si[per-mode=fraction]{\watt\per\metre\squared\per\kelvin}  & 3200kg &2.7 \si[per-mode=fraction]{\mega\joule\per\kelvin}
%	\\ %\hline
%	mennyezet & 16 \si{\metre\squared} & 5 \si[per-mode=fraction]{\watt\per\metre\squared\per\kelvin} & 3200kg &2.7 \si[per-mode=fraction]{\mega\joule\per\kelvin}	
	\\ \hline

%	belső válaszfalak & 50 \si{\metre\squared} & 7 & 50*100kg & 50*100*840		
%	\\ \hline
%	11 & Internal limit active
%	\\ \hline
%	12-13 & Operation mode specific
%	\\ \hline
%	14-15 & Reserved
\end{tabu}

	\label{tab:HeatExchangeTypes}
	%
	%\label{tab:TabularExample}
	%\tabref{TabularExample}~táblázat
\end{table}
 


 
 
 
 
 

\subsection*{Hőátbocsátási tényező számítása}




\subsection*{Hőtároló képesség}

\begin{equation}\label{eq_hotarolo}
\begin{aligned}
Q_{ki,fal} &= U_{fal}A_{fal}\Delta T = 200\si{\watt}\\[10pt]
Q_{ki,ablak} &= U_{ablak}A_{ablak}\Delta T = 400\si{\watt}
\end{aligned}
\end{equation}

% \vspace{24pt}

% Falszerkezet, ablakszerkezet

\section{A megvalósított modell}

Figyelembe kell vennem a ház hőveszteségeit és hőtároló képességét is, a () és () egyenletek alapján, melynek paraméterei a \ref{table_house_parameters}. táblázatban találhatók.
%Ennek paraméterei: a határoló elemek felszíne, hőátbocsátási tényezője, a hőtároló elemek fajhője.
Az alábbi táblázat értékeinek nagy részét ki lehet tölteni a tanúsítványból.
Az épület hőigénye numerikusan is szerepel, ám ez pl. éves átlagolással adódik, nem csak a fűtési rendszert, hanem a várható időjárást is figyelembe veszi, illetve az energiaigénynél nem csak a fűtési, hanem használati melegvíz előállítására felhasznált energiát is.

A Matlab Simscape model és Lapusan\footnote{Development of a Multi-Room Building Thermodynamic Model Using Simscape Library - Ciprian Lapusan} hőátadásnál a réteges szerkezetekben számolt konvekcióval és kondukcióval is. Viszont ezek az adatok egyben is kezelhetők, a követelményeket ezekre a költségoptimalizált követelményszint\footnote{ A  költségoptimalizált követelményszintek megtalálhatók a 7/2006. rendelet \cite{TNM2006} 5. mellékletében.} adja meg. Régebbi épületek ezt a szintet nem tudják teljesíteni, ezekre jellemző értékeket adtam meg az alábbi táblázatban. 


Ám nem szabad összekeverni az U értéket (hőátbocsátási tényező) és az $\alpha_i$ konvektív hőátadási tényezőt, amit a válaszfalakra, padlóra és mennyezetre adtam meg, hiszen ezeken a modell szerint a helyiség nem veszt hőt, csak a hőtároló elemeknek adódik át. Viszont itt célszerű lenne a konvektív hőátadást is beleszámolni. 


%(Baromi érdekes, hogy nálunk otthon van egy hőcserélő a használati melegvíz és a fűtési rendszer között, annak a hatásait is lehetne nézni.)

%\begin{table}[ht]
%	\footnotesize
%	\centering
%	\caption{Az órajel-generátor chip órajel-kimenetei.} \label{tab:SysClocks}
%	\begin{tabular}{ | l | c | c |}
%		\hline
%		Órajel & Frekvencia & Cél pin \\ \hline
%		CLKA & 100 MHz & FPGA CLK0\\
%		CLKB & 48 MHz  & FPGA CLK1\\
%		CLKC & 20 MHz  & Processzor\\
%		CLKD & 25 MHz  & Ethernet chip \\
%		CLKE & 72 MHz  & FPGA CLK2\\
%		XBUF & 20 MHz  & FPGA CLK3\\
%		\hline
%	\end{tabular}
%	\label{tab:TabularExample}
%\end{table}





\begin{table}[H]
	\footnotesize
	\centering
	\caption{Egy szoba határoló felületei és azok termikus tulajdonságai}
	\renewcommand{\arraystretch}{2} % to increase cell height
	\taburulecolor{gray}
	
	%\begin{tabular}{|p{0.8cm}|p{1cm}|p{1cm}|p{1cm}|p{1cm}|p{1cm}|p{1cm}|p{1cm}|}
	
	\newcolumntype{C}[1]{>{\centering\arraybackslash}p{#1}}
\newcolumntype{R}[1]{>{\raggedleft\arraybackslash}p{#1}}

\begin{tabu}{@{}p{3.5cm}p{1.2cm}p{2cm}p{3cm}p{3cm}@{}}
	%\begin{tabu}{|p{3cm}|p{1.2cm}|p{2cm}|p{3cm}|p{3cm}|}
	%{p{1.5cm}|C{0.8cm}|C{0.8cm}|C{0.8cm}|C{0.8cm}|C{0.8cm}|C{0.8cm}|C{0.8cm}|C{0.8cm}|}
	%\multicolumn{1}{l}{}&\multicolumn{8}{l}{SDO header (első adatbyte) - master kérése}
	%\\ 		\cline{2-9}\cline{2-9}

	veszteséges elemek& méret & $U$    & hőtároló tömeg & hőkapacitás
	\\ \hline%\hhline{=====}
	külső fal & 4.5 \si{\metre\squared} & 2 \si[per-mode=fraction]{\watt\per\metre\squared\per\kelvin} & 900kg & 756 \si[per-mode=fraction]{\kilo\joule\per\kelvin}
	\\ %\hline
	ablak & 4 \si{\metre\squared} & 4 \si[per-mode=fraction]{\watt\per\metre\squared\per\kelvin} & - & -
	\\ %\hline
\end{tabu}
\vspace*{10pt}
\begin{tabu}{@{}p{3.5cm}p{1.2cm}p{2cm}p{3cm}p{3cm}@{}}
	csak hőtároló elemek & méret & $h_t$    & hőtároló tömeg & hőkapacitás \\	\hline%\hhline{=====}
	belső válaszfalak & 50 \si{\metre\squared} & 7 \si[per-mode=fraction]{\watt\per\metre\squared\per\kelvin} & 5000kg & 4.2 \si[per-mode=fraction]{\mega\joule\per\kelvin}	
	\\ %\hline
	padló & 16 \si{\metre\squared} & 11 \si[per-mode=fraction]{\watt\per\metre\squared\per\kelvin}  & 3200kg &2.7 \si[per-mode=fraction]{\mega\joule\per\kelvin}
	\\ %\hline
	mennyezet & 16 \si{\metre\squared} & 5 \si[per-mode=fraction]{\watt\per\metre\squared\per\kelvin} & 3200kg &2.7 \si[per-mode=fraction]{\mega\joule\per\kelvin}	
	\\ %\hline

%	belső válaszfalak & 50 \si{\metre\squared} & 7 & 50*100kg & 50*100*840		
%	\\ \hline
%	11 & Internal limit active
%	\\ \hline
%	12-13 & Operation mode specific
%	\\ \hline
%	14-15 & Reserved
\end{tabu}

	

	\label{table_house_parameters}
	
	%\label{tab:TabularExample}
	%\tabref{TabularExample}~táblázat
	
\end{table}

 A példában a Schönherz Zoltán kollégium egy szobájának megfelelő méretű helyiséget vettem fel. Minden szobának van ablaka és külső fala, egy átlagos szobát 4 másik vesz körül. A belső falakon nem veszt hőt, csak az ablakon ill. a külső falon. Feltételezzük, hogy a radiátoros fűtést egy szeleppel szabályozhatjuk, amit tetszőleges mértékben nyithatunk ki.
 A napsütés hőnyereségét is figyelembe vehetjük.%, úgy, hogy egy hőforrás a padlót melegíti.

%A modell mintavételi ideje?
%A teljesítményeket megnöveljük és semmi mást, az nem lesz ekvivalens. 

\subsubsection*{Hőigény:}

A külső falon

\begin{equation}\label{eq_hoigeny}
\begin{aligned}
		Q_{ki,fal} &= U_{fal}A_{fal}\Delta T = 200\si{\watt}\\[10pt]
		Q_{ki,ablak} &= U_{ablak}A_{ablak}\Delta T = 400\si{\watt}
\end{aligned}
\end{equation}

Amennyiben a méretezési hőmérséklet $\Delta T=$ \SI{-2}{\celsius}, ami a téli átlaghőmérséklet Magyarországon.\footnote{Épületfizika kurzus alapján vettem az átlaghőmérsékletet \SI{-2}{\celsius}-nak.}


%(Gondolatkísérlet: HA nem hatna zavarás, csak az időállandók számítanának, a pontos teljesítményveszteségek, nyereségek nem. Azaz mindegy volna hogy 1000W hő szökik ki és ehhez tartozik 1500W-nyi fűtési kapacitás, vagy hogy 5000W és 7500W ezek az értékek. Ám pl. napsütés hatásakor nem csak az arányok hanem a konkrét teljesítmények is kellenek...

%Így a modell egyik belső változója bizonyosan a teljesítmény kell, hogy legyen. Erre a belső változóra hat majd zavarás: emberek jelenléte kb. \SI{80}{\watt} 1 főre, napsütés, szellőztetés, stb.)

%\hrulefill


%Erre ki kellene számítani a hőigényt, figyelembe véve azt hogy mennyi hő szökik el a külső és belső határoló felületeken keresztül.
%A gyakorlati alkalmazásokban szeretnék majd az energetikai tanúsítványból kiindulni.%, így gyakorlatilag a szoba energetikai tanúsítását végzem el - olyan szinten, amennyire nekem szükséges.


%Ashrae HVAC - 6.19 Panel H \& C. - Controls strategy
%
%A modellt a jellemző szerkezeti tulajdonságok alapján írtam fel (indoklás a táblázathoz). A modellezés Gouda alapján történik, gyakorlatilag csomóponti egyenleteket kell felírni az alábbi hálózatra, amiben az ellenállások a rétegrendi hőátbocsátási tényező reciprokai. A hőtároló képességeket kapacitások modellezik. Ezeket az elemeket Simscape-ben implementáltam, a hőáramok így áttekinthetők és a paraméterek könnyen változtathatók.
%
%A ház modelljének felírásakor figyelembe vettem a hőtároló elemeket. A pontos (reális) modell felállításakor ezek hőtartalmát (a hőáram integrálja egyensúlyi állapotban legyen 0, azaz egy nagyobb ciklusban a felvett és leadott hője egyenlő) az egyensúlyi állapothoz közelinek vettem.
%
%Viszont a szabályzótervezéshez identifikálni kell, ekkor pedig a falak, ill. szoba levegőjének kezdeti állapotát (hőmérsékletét) azonosnak vettem a külső hőmérséklettel. Így ha a hőkülönbség a modell kimenő jele, akkor lineáris a rendszer: 0 bemenetre (fűtés) 0 kimenetet ad.

%\subsection{Megvalósítás MATLAB-ban}

%a simscape elemek kapcsolatai

\section{Fűtési rendszer és ház kapcsolata}

Ha a szabályzást egy már meglévő épületre tervezzük, akkor csak a rendszerek adatait kell felvenni, illetve identifikálni. A szakdolgozatban tárgyalt egyszerű példa során csak egy részét ismerem a paramétereknek, tehát méretezési kérdéseket is fogok érinteni.  Szerencsére az új építésű házaknál kötelező energetikai tanúsítás\footnote{TNM 2006 rendelet alapján kötelező az energetikai tanúsítvány pl. \textit{átlagos} lakóépületekre, irodákra.} egy meglehetősen részletes lajstromot ad az épület hőtechnikai tulajdonságairól. Ez alapján lehet egy hozzávetőlegesen jó modellünk az épületről, illetve a fűtési rendszerről is találhatók adatok paraméterek. Az interneten számos tanúsító cég töltött fel minta tanúsítványokat, amiben a számítások levezetése, indoklása is megtalálható. Így az energetikai tanúsítvány lehet egy interface a szakdolgozatban bemutatott modell és a gyakorlati alkalmazások között: valódi épület tanúsítványa alapján a modellem paraméterezhető.


Amikor a fűtési rendszer viselkedését szimulálom, nekem kell megalkotni mind a szabályzott épületrész, mind a fűtési rendszer modelljét. Így tehát ez a modellezésen felül egy méretezési feladat is, amit egy kész épületnél már elvégeztek a tervezés során, és a megfelelő fűtési teljesítmény áll rendelkezésre. %illeszkedik az igényekhez és a körülményekhez.






%
%\section{Alkalmazott fűtési rendszerek}
%
%Az alkalmazott fűtési rendszerek az épületet annak különböző pontjain gerjesztik. (Belső változóira nem egyformán hatnak: a kimeneten a változás intenzitása és sebessége más-más.) A teljes plant modell a fűtési rendszer és a ház sorba kötésével adódik.
%
%A kettő között az interface az, hogy hol avatkozunk be. Így a ház bemenetei igazából a belső változókra vonatkozó "zavarások" (a külső hőmérséklethez képest)

%\section{A modell átviteli függvénye}
%A Simulinkben identifikáltam, aztán az adatokat a sys ident toolbox-szal dolgoztam fel, tudva a modell struktúráját. (az átviteli fv. számlálójának, nevezőjének a fokszámait)

%\section{TABS}


\pagebreak
%\hrulefill