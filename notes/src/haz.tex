\section{Ház modellje}

\subsection{Big picture}

A modellalkotásnál igyekszek energetikai tanúsítványban szereplő adatokat felhasználni.
Figyelembe kell vennem a ház hőveszteségeit és hőtároló képességét is.

A kinyerhető adatok: a határoló elemek felszíne,

hőigény numerikusan is szerepel
A Simscape-ben hőátadási tényezőket és hőtároló tömegeket vettem fel.


\section{A felírt modell}

A schönherzes kollégiumi szoba határoló elemeit vettem fel. Ez \SI{80}{\meter\squared}-nyi belső falfelület, 4

Táblázatban össze kellene foglalni.

\begin{table}[H]
	\centering
	
	\renewcommand{\arraystretch}{2} % to increase cell height
	\taburulecolor{gray}
	
	%\begin{tabular}{|p{0.8cm}|p{1cm}|p{1cm}|p{1cm}|p{1cm}|p{1cm}|p{1cm}|p{1cm}|}
	
	\newcolumntype{C}[1]{>{\centering\arraybackslash}p{#1}}
\newcolumntype{R}[1]{>{\raggedleft\arraybackslash}p{#1}}

\begin{tabu}{@{}p{3.5cm}p{1.2cm}p{2cm}p{3cm}p{3cm}@{}}
	%\begin{tabu}{|p{3cm}|p{1.2cm}|p{2cm}|p{3cm}|p{3cm}|}
	%{p{1.5cm}|C{0.8cm}|C{0.8cm}|C{0.8cm}|C{0.8cm}|C{0.8cm}|C{0.8cm}|C{0.8cm}|C{0.8cm}|}
	%\multicolumn{1}{l}{}&\multicolumn{8}{l}{SDO header (első adatbyte) - master kérése}
	%\\ 		\cline{2-9}\cline{2-9}

	veszteséges elemek& méret & $U$    & hőtároló tömeg & hőkapacitás
	\\ \hline%\hhline{=====}
	külső fal & 4.5 \si{\metre\squared} & 2 \si[per-mode=fraction]{\watt\per\metre\squared\per\kelvin} & 900kg & 756 \si[per-mode=fraction]{\kilo\joule\per\kelvin}
	\\ %\hline
	ablak & 4 \si{\metre\squared} & 4 \si[per-mode=fraction]{\watt\per\metre\squared\per\kelvin} & - & -
	\\ %\hline
\end{tabu}
\vspace*{10pt}
\begin{tabu}{@{}p{3.5cm}p{1.2cm}p{2cm}p{3cm}p{3cm}@{}}
	csak hőtároló elemek & méret & $h_t$    & hőtároló tömeg & hőkapacitás \\	\hline%\hhline{=====}
	belső válaszfalak & 50 \si{\metre\squared} & 7 \si[per-mode=fraction]{\watt\per\metre\squared\per\kelvin} & 5000kg & 4.2 \si[per-mode=fraction]{\mega\joule\per\kelvin}	
	\\ %\hline
	padló & 16 \si{\metre\squared} & 11 \si[per-mode=fraction]{\watt\per\metre\squared\per\kelvin}  & 3200kg &2.7 \si[per-mode=fraction]{\mega\joule\per\kelvin}
	\\ %\hline
	mennyezet & 16 \si{\metre\squared} & 5 \si[per-mode=fraction]{\watt\per\metre\squared\per\kelvin} & 3200kg &2.7 \si[per-mode=fraction]{\mega\joule\per\kelvin}	
	\\ %\hline

%	belső válaszfalak & 50 \si{\metre\squared} & 7 & 50*100kg & 50*100*840		
%	\\ \hline
%	11 & Internal limit active
%	\\ \hline
%	12-13 & Operation mode specific
%	\\ \hline
%	14-15 & Reserved
\end{tabu}

	
	\caption{Különböző SDO típusok felépítése - minden adat hexában értendő}
	\label{table-sdotypes}
\end{table}




\subsection{Fűtési rendszer és ház kapcsolata}





A fűtési rendszer és a szabályzás alapvető validálásához egyszerű házmodelleket fogok felállítani.

Szinte a legegyszerűbb ilyen tekintetben egy kollégiumi szoba modellje. Egy átlagos szobát 4 másik vesz körül, van ablaka, egy radiátora.
Erre ki kellene számítani a hőigényt, figyelembe véve azt hogy mennyi hő szökik el a külső és belső határoló felületeken keresztül.
A gyakorlati alkalmazásokban szeretnék majd az energetikai tanúsítványból kiindulni, így gyakorlatilag a szoba energetikai tanúsítását végzem el - olyan szinten, amennyire nekem szükséges.


Ashrae HVAC - 6.19 Panel H \& C. - Controls strategy

Ezért utánanéztem a jellemző szerkezeti tulajdonságoknak. A modellezés Gouda alapján történik, gyakorlatilag csomóponti egyenleteket kell felírni az alábbi hálózatra, amiben az ellenállások a rétegrendi hőátbocsátási tényező reciprokai. A hőtároló képességeket kapacitások modellezik.

A ház modelljének felírásakor figyelembe vettem a hőtároló elemeket. A pontos (reális) modell felállításakor ezek hőtartalmát (a hőáram integrálja egyensúlyi állapotban legyen 0, azaz egy nagyobb ciklusban a felvett és leadott hője egyenlő) az egyensúlyi állapothoz közelinek vettem.

Viszont a szabályzótervezéshez identifikálni kell, ekkor pedig a falak, ill. szoba levegőjének kezdeti állapotát (hőmérsékletét) azonosnak vettem a külső hőmérséklettel. Így ha a hőkülönbség a modell kimenő jele, akkor lineáris a rendszer: 0 bemenetre (fűtés) 0 kimenetet ad.

Az identifikációnál így minden hőtároló elem (levegő, külső és belső falak) kezdeti hőmérsékletét a környezeti hőmérséklettel azonosnak vettem.
(Még egy mérés lehet az, ha mérhető zavarásunk van, pl. kezdeti \SI{20}{\celsius} hőmérsékletet beállítva a környezeti hőmérséklet zuhan 20-ról \SI{10}{\celsius}-ra. )



\pagebreak