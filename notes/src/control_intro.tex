\section{Szabályzó kiválasztása és tervezése}


Az identifikált modellekre többféle szabályzót tervezek, illetve próbálok ki.

\subsection{Ismerkedés az MPC szabályzással}

Ahhoz, hogy az MPC szabályzás működését, tulajdonságait meg tudjam figyelni, lépésről lépésre fogok feature-öket hozzáadni.

A kezdeti cél egy "sima" szabályzás. Kérdés, hogy egyáltalán tud-e ilyet az MPC. Gyanítom, hogy a hibaminimalizáló függvény megfelelő megadásával tud: ha egy négyzetes hibaminimalizáló van rajta, \textit{biztosan "jó"} lesz.\footnote{Bármit is jelentsen a \textit{jó} szabályzás.}

\begin{formal}
	Plant bemenetek lehetnek:
	\begin{itemize}[noitemsep,topsep=-8pt,parsep=0pt,partopsep=0pt]
		\item kazán bekapcsolása
		\item előremenő hőmérséklet - unmeasured VAGY uncontrolled inputként
		\item 1 db. fűtőtest (most radiátor) szelepének tömegárama (szelep áteresztése)
		\item Később több fűtőtest vagy többféle fűtőtestek (padlófűtés, különböző teljesítményű radiátorok) szabályozása
		\item környezeti hőmérséklet: mért bemenet - később predikció is lesz rá. Hatása a kimeneten már identifikálva lett, 3 pólussal és 2 zérussal tökéletesen lekövethető.
		\item napsugárzás zavaró hatása - szimulálható  a bizonytalansága valószínűleg nagy lesz
	\end{itemize}

	Belső változók - fűtési rendszer és ház kapcsolata
	\begin{itemize}[noitemsep,topsep=-6pt,parsep=0pt,partopsep=0pt]
			\item napsugárzás - radiatív, az ablak felületével és a szöggel arányos
			\item fűtőtestek sugárzó és konvektív hőárama
	\end{itemize}

	Paraméterek a plantben nem állandók:
	\begin{itemize}[noitemsep,topsep=-6pt,parsep=0pt,partopsep=0pt]
		\item hőátadási tényezők hőmérsékletfüggők, áramlási sebesség-függők (szél)
		\item szellőztetés, belső hőterhelés hatása
	\end{itemize}
\end{formal}

Az elvárás a következő lépésben az, hogy ha egy $t_0$ időpontban a rendszer egy adott állapotban van, és várható egy zavarás $\Delta t$ idő múlva (vagy mértem egy zavarást MOST és a hatása csak később jelenne meg a kimeneten), akkor a rendszer megfelelően beavatkozzon.

(Azaz ha fél óra múlva \SI{10}{\celsius}-al melegebb lesz, ne fűtsön.)

\subsection{Offline MPC - supervisory control}

\textit{4.4. Approaches without real-time dynamic optimization}\footnote{Thieblemont-ból. A real-time update nélküli MPC a legegyszerűbb és a leggyorsabban kiszámolható. Gyakran más irányítási technikákon alapul.} Döntési fa, affin leképezés ilyenek.

Elkészíteni az offline döntési hálót viszont nehezebb.

\subsection{A MATLAB MPC Toolbox elemei}
Az MPC blokknak van egy alapértelmezett költségfüggvénye, és ennek a súlyozását lehet beállítani.
Külön beállítható a szabályzási és a szimulációs horizont.
Ezek optimális beállításai 

A kezdeti MPCkontrollert egyszerűen étre lehet hozni az identifikált modellből és a bemenetek típusának megadásával. (A szelep a beavatkozó jelem, ebből származik a kazán bekapcsolása (hiszterézises cucc), illetve a plantnek van még egy bemenete, egy mérhető zavarás.) Ezután a bemenetek értékkészletét adtam meg, illetve van egy normalizáló faktor, ami a jellemző\textit{full scale}.

Az optimalizálás egy költségfüggvény minimalizálását jelenti, amiben \textit{büntetjük} a referenciajeltől való eltérést és a beavatkozó jelek \textbf{értékét vagy változását}.

A fenti a klasszikus MPC, tov. info. Baochang DING, Modern MPC című könyvében olvasható.
 
\subsubsection{Módosítások az MPC-ben}

A súlyozástmódosítva adhatunk költséget a beavatkozásnak, csökkentve így pl. annak a frekvenciáját. Ez a referenciakövetést rontja, de esetünkben nem cél a tized \si{\celsius}-os pontosság, hanem az energiamegtakarítás.
Pontosan fel kellene ezért írni a forintosított költségét a beavatkozásnak, és ezt minimalizálni.

\subsection{Validálás}
Szimulációval ellenőrizzük a szabályzás robosztusságát. Ehhez megnöveltem a hőtároló tömegeket.

Ötlet: random időpontban lehetne ablaknyitást szimulálni.
Napsütés hatásmechanizmusa.
Radiant heat transfer paramétere továbbra sem olyan világos: sok publikációban a hőmérsékletkülönbség lineáris függését tartalmazza és nem a Stefan-Boltzmann törvény szerinti negyedik hatvány szerintit