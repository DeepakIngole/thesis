\chapter{Szabályzó kiválasztása és analízise}


Az identifikált modellekre többféle szabályzót tervezek, illetve próbálok ki.

\section{Ismerkedés az MPC szabályzással}

Ahhoz, hogy az MPC szabályzás működését, tulajdonságait meg tudjam figyelni, lépésről lépésre fogok feature-öket hozzáadni.

A kezdeti cél egy "sima" szabályzás. Kérdés, hogy egyáltalán tud-e ilyet az MPC. Gyanítom, hogy a hibaminimalizáló függvény megfelelő megadásával tud: ha egy négyzetes hibaminimalizáló van rajta, \textit{biztosan "jó"} lesz.\footnote{Bármit is jelentsen a \textit{jó} szabályzás.}

\begin{formal}
	Plant bemenetek lehetnek:
	\begin{itemize}[noitemsep,topsep=-8pt,parsep=0pt,partopsep=0pt]
		\item kazán bekapcsolása
		\item előremenő hőmérséklet - unmeasured VAGY uncontrolled inputként
		\item 1 db. fűtőtest (most radiátor) szelepének tömegárama (szelep áteresztése)
		\item Később több fűtőtest vagy többféle fűtőtestek (padlófűtés, különböző teljesítményű radiátorok) szabályozása
		\item környezeti hőmérséklet: mért bemenet - később predikció is lesz rá. Hatása a kimeneten már identifikálva lett, 3 pólussal és 2 zérussal tökéletesen lekövethető.
		\item napsugárzás zavaró hatása - szimulálható  a bizonytalansága valószínűleg nagy lesz
	\end{itemize}

	Belső változók - fűtési rendszer és ház kapcsolata
	\begin{itemize}[noitemsep,topsep=-6pt,parsep=0pt,partopsep=0pt]
			\item napsugárzás - radiatív, az ablak felületével és a szöggel arányos
			\item fűtőtestek sugárzó és konvektív hőárama
	\end{itemize}

	Paraméterek a plantben nem állandók:
	\begin{itemize}[noitemsep,topsep=-6pt,parsep=0pt,partopsep=0pt]
		\item hőátadási tényezők hőmérsékletfüggők, áramlási sebesség-függők (szél)
		\item szellőztetés, belső hőterhelés hatása
	\end{itemize}
\end{formal}

Az elvárás a következő lépésben az, hogy ha egy $t_0$ időpontban a rendszer egy adott állapotban van, és várható egy zavarás $\Delta t$ idő múlva (vagy mértem egy zavarást MOST és a hatása csak később jelenne meg a kimeneten), akkor a rendszer megfelelően beavatkozzon.

(Azaz ha fél óra múlva \SI{10}{\celsius}-al melegebb lesz, ne fűtsön.)

\subsection{A MATLAB MPC Toolbox elemei}
Az MPC blokknak van egy alapértelmezett költségfüggvénye, és ennek a súlyozását lehet beállítani.
Külön beállítható a szabályzási és a szimulációs horizont.
Ezek optimális beállításai 

A kezdeti MPC szabályzót egyszerűen létre lehet hozni az identifikált modellből és a bemenetek típusának megadásával. (A szelep a beavatkozó jel, illetve a plantnek van még egy bemenete, egy mérhető zavarás.) Ezután a bemenetek értékkészletét adtam meg, illetve van egy normalizáló faktor, ami a jellemző\textit{full scale}.

Az optimalizálás egy költségfüggvény minimalizálását jelenti, amiben \textit{büntetjük} a referenciajeltől való eltérést és a beavatkozó jelek \textbf{értékét vagy változását}.

A fenti a klasszikus MPC, tov. info. Baochang DING, Modern MPC című könyvében olvasható.


\begin{lstlisting}[
style=Matlab-editor,
basicstyle=\mlttfamily,
escapechar=`,
]
tf_19_toMPC=setmpcsignals(tf19, 'Manipulated',[2 3],'MeasuredDisturbances' ,1)

tf_19_toMPC =

From input "u1" to output "y1":
   3.776e-05 s^2 + 5.958e-09 s + 8.529e-15
--------------------------------------------
s^3 + 0.002341 s^2 + 9.301e-09 s + 8.241e-15


From input "u2" to output "y1":
                 7.74 s + 0.000236
---------------------------------------------------
s^4 + 1.269e04 s^3 + 4262 s^2 + 3.299 s + 8.454e-06


From input "u3" to output "y1":
  6.361e-05
-------------
s + 2.637e-06

Input groups:
       Name                 Channels
   Manipulated                 2,3
     Measured                   1

Output groups:
       Name                 Channels
     Measured                   1

Name: tf19
Continuous-time identified transfer function.
Parameterization:
    Number of poles: [3 4 1] Number of zeros: [2 1 0]
    Number of free coefficients: 14
    Use "tfdata", "getpvec", "getcov" for parameters and their uncertainties.

Status:
Estimated using TFEST on time domain data "tf_3in1out__68d".
Fit to estimation data: 82.28% (stability enforced)
FPE: 1.707, MSE: 1.706 


>> mpc_control_slab=mpc(tf_19_toMPC,1)
-->Converting linear model from System Identification Toolbox to statespace.
-->The "PredictionHorizon" property of "mpc" object is empty. Trying
PredictionHorizon = 10.
-->The "ControlHorizon" property of the "mpc" object is empty. Assuming 2.
-->The "Weights.ManipulatedVariables" property of "mpc" object is empty.
Assuming default 0.00000.
-->The "Weights.ManipulatedVariablesRate" property of "mpc" object is
empty. Assuming default 0.10000.
-->The "Weights.OutputVariables" property of "mpc" object is empty.
Assuming default 1.00000.
MPC object (created on 30-Oct-2018 20:51:50):
---------------------------------------------
Sampling time: 1 (seconds)
Prediction Horizon: 10
Control Horizon: 2
Plant Model: 
MATLAB Command Window Page 6
--------------
2 manipulated variable(s) -->| 8 states |
| |--> 1 measured output
(s)
1 measured disturbance(s) -->| 3 inputs |
| |--> 0 unmeasured
output(s)
0 unmeasured disturbance(s) -->| 1 outputs |
--------------
Indices:
(input vector) Manipulated variables: [2 3 ]
Measured disturbances: [1 ]
(output vector) Measured outputs: [1 ]
Disturbance and Noise Models:
Output disturbance model: default (type "getoutdist
(mpc_control_slab)" for details)
Measurement noise model: default (unity gain after scaling)
Weights:
ManipulatedVariables: [0 0]
ManipulatedVariablesRate: [0.1000 0.1000]
OutputVariables: 1
ECR: 100000
State Estimation: Default Kalman Filter (type "getEstimator
(mpc_control_slab)" for details)
Unconstrained
>> mpc_control_slab.ManipulatedVariables(1).Min = 0;
>> mpc_control_slab.ManipulatedVariables(2).Min = 0;
>> mpc_control_slab.ManipulatedVariables(2).Max = 1;
mpc_control_slab.ManipulatedVariables(1).Max = 1;

\end{lstlisting}


\subsection{Az automatikusan létrehozott MPC tulajdonságai}

Az MPC szabályzót létrehoztam a toolbox-szal, az identfikált szakaszból. Beállítottam a be-és kimenetek jellegét, korlátait. A ki-és bemeneteket helyesen bekötve már működött is a szabályzás.



\subsubsection{A kezdeti szabályzó problémái}
Igaz, hogy az alapjelkövetés gyakorlatilag tökéletes volt, de a beavatkozó jelnek a gyakorlatban nem csak a nagysága, hanem a frekvenciája is korlátos. Ezért a beavatkozó szervnek is kell egy átviteli függvény ideális esetben. (Itt most a szelepről van szó.)

A \textit{súlyozatlan} MPC nem vette figyelembe a beavatkozójel változásának \textit{nagy} költségét, ezért irreálisan gyorsan nyitotta és zárta azt.
A gyakorlatban nincs szükség tűpontos referenciakövetésre, a hőmérséklet kb. \SI{1}{\celsius}-ot ingadozhat. ($\pm$ \SI{0.5}{\celsius}) Ha ezt megengedjük, a beavatkozás költsége lecsökkenhet.

\subsubsection{Robosztusság}

A Simulinkben identifikált modellre pontosan lehetett átviteli függvényt illeszteni, így a szabályzóban futó modell gyakorlatilag tökéletes volt. Gyakorlatban viszont a modellek igencsak pontatlanok lehettek, így megvizsgáltam a szabályzás viselkedését megváltozott paraméterekkel is. Ezt a szabályzás alapvetően jól viselte, a referenciakövetés minősége megmaradt.

\section{A szabályzó paramétereinek finomítása, hangolása, alapbeállítások felülírása}

A mintavételi időt megnöveltem. A ház identifikációját és az MPC tervezést is 5 perces időállandóval végeztem. A lépéseket először egy unit test részben hajtottam végre.

\begin{itemize}[noitemsep,topsep=-8pt,parsep=0pt,partopsep=0pt]
	\item A mintavételi idő növelése a Matlab default workspace-ben magával vonja, hogy a Simulink blokkban is módosul a $T_s$. 
	\item A Simulinkben az időt a jobb alsó sarokban mindig mp-ben írja ki. Ámde ha a steppingnél 1000 step-et állítok be, az a jobb alsó sarokban $T_s$-sel felskálázva fogja a mp-t mutatni. Azaz 5 perces sampling time esetén 1 step a jobb alsó sarokban T=300 mp-nek felel meg.
	\item A mintavételi idő megválasztása nagyban meghatározza a költségfüggvény értékét.
	\item
	\item
	\item
	\item
	\item
	\item	
\end{itemize}



\subsubsection{Módosítások az MPC-ben}

A súlyozást módosítva adhatunk költséget a beavatkozásnak, csökkentve így pl. annak a frekvenciáját. Ez a referenciakövetést rontja, de esetünkben nem cél a tized \si{\celsius}-os pontosság, hanem az energiamegtakarítás.
Pontosan fel kellene ezért írni a forintosított költségét a beavatkozásnak, és ezt minimalizálni.

Egyensúlyt kell találni a referenciakövetés és a beavatkozás között. Külön érdekesség, hogy ha nem távfűtést használunk, akkor a kis beavatkozásnak is nagy költsége van. Erre a súlyozásnál egy LUTot lehetne használni. Btw. a hőszivattyúk kis terhelésen is nagy hatásfokkal működnek. Online weight tune elképzelhető, pl. a beavatkozó jeltől függően.

\subsection{Az MPC költségfüggvénye}




Nem csak a bemenetek értékei súlyozhatók. Az egyik kinyomtatott doksiban nem csak a bemenetek, vagy a hibajel kap súlyozást, hanem a villamos energia aktuális ára is tényező.

Kell keresni egy suitable költségfüggvényt. Illetve megfontolandó lenne vízhőmérsékletre szabályozni, annak a költsége szemléletesebb.

\subsubsection{Súlyozás}
A beavatkozó jelek és a szakasz kimenete is súlyozható, hogy azok a költségfüggvénybe mennyire szóljanak bele. A MATLAB lehetőséget ad arra, hogy ezeket a súlyokat működés közben befolyásoljuk. A Simulinkben beállítottam, hogy a radiátor szelepének alacsony kimenetére a szelep súlya 1 legyen, viszont 30\%-ban kinyitott szelepre csökkenjen le 0.5-re. Ez nem hozott javulást, ugyanis a nagy súllyal az MPC a predikciós horizonton végrehajtott egy optimalizálást. Ám ha a szelepet kinyitotta, a súlyok megváltoztak, így az optimális költségű beavatkozójel is. Viszont ennek éppen elősegítenie kellett volna a szabályzást, ehelyett összezavarta.


Valójában fordítva kell. Kis amplitúdó esetén NULLA pluszköltség még jobban kinyitni ("Szívesen" növekedjen tovább ha még csak kicsit van nyitva.) Csak ha félig van kinyitva, akkor növeljük a költséget.

Sajnos viszont a fenti költségeket nem lehet (nehéz) megfeleltetni forintosított tételeknek.

\hrulefill

\subsection{Offline MPC - supervisory control}

\textit{4.4. Approaches without real-time dynamic optimization}\footnote{Thieblemont-ból. A real-time update nélküli MPC a legegyszerűbb és a leggyorsabban kiszámolható. Gyakran más irányítási technikákon alapul.} Döntési fa, affin leképezés ilyenek.

Elkészíteni az offline döntési hálót viszont nehezebb.


	


\subsection{Validálás}
Szimulációval ellenőrizzük a szabályzás robosztusságát. Ehhez megnöveltem a hőtároló tömegeket.

Ötlet: random időpontban lehetne ablaknyitást szimulálni.
Napsütés hatásmechanizmusa.
Radiant heat transfer paramétere továbbra sem olyan világos: sok publikációban a hőmérsékletkülönbség lineáris függését tartalmazza és nem a Stefan-Boltzmann törvény szerinti negyedik hatvány szerintit