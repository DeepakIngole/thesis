\documentclass[14pt]{beamer}
%aspectratio=169
\usepackage[utf8]{inputenc}
\usepackage[T1]{fontenc}
\usepackage[magyar]{babel}
\usetheme{default}
\usepackage{subcaption} %subfigure

\begin{document}

\begin{frame}{Keywords}

\end{frame}


\begin{frame}{Műszaki tartalom}

Piacvezérelt vagy tudásalapú terméket szeretnék?

\begin{itemize}
	\setlength{\itemsep}{7pt}
	\item Modellalapú szabályozás (optimális, prediktív)
	\begin{itemize}
		\item nagyobb komfort, alacsonyabb költségek
		\item innovatív, kutatják, publikálják az eredményeket
		\item komplex modellek, MIMO rendszerek kezelése
		\item optimalizációra visszavezethető beavatkozás\footnote{Az optimális beavatkozásnak sokféle kritériuma lehet.}
		%\item fontosabb a szabályozás minősége, mint hogy univerzális legyen
		%\item mérhető és nem mérhető zavarások figyelembe vétele
	\end{itemize}
	\item Kiindulás a piacon elérhető megoldásokból
	\begin{itemize}
		\item felkapott: intelligens otthon rendszerek (marketing)
		\item multicégek termékei: Siemens, Bosch, Johnson Controls, Honeywell, Danfoss termosztátjai, okos rendszerei (max. egy PI-szabályozó)
		%\item fontosabb a szabályozás minősége, mint hogy univerzális legyen
		%\item mérhető és nem mérhető zavarások figyelembe vétele
	\end{itemize}
\end{itemize}
\end{frame}

\end{document}