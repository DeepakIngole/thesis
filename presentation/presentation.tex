\documentclass[14pt,handout]{beamer}
%aspectratio=169
\usepackage[utf8]{inputenc}
\usepackage[T1]{fontenc}
\usepackage[magyar]{babel}
\usetheme{default}
%\usepackage{enumitem} %settings of itemize, e.g. itemsep


\title{Korszerű fűtési rendszerek szabályozása}
\author{Gyulai László}
%\institute{Szakdolgozat bemutatás}
\date{2019. január 7.}



\begin{document}
	
	\frame{\titlepage}
\begin{frame}{Bevezető}

%[shrink=-25]
    \begin{itemize}
        \item Témaválasztás szempontjai
        \pause
        \setlength{\itemsep}{6pt}
        \begin{itemize}
            \item szabályozástechnikai vonatkozás
            \item gyakorlati haszon, piaci igény 
        \end{itemize}
    	\pause
    	\item \underline{Korszerű fűtési rendszerek szabályozása}
    	\pause
        \begin{itemize}
            \setlength{\itemsep}{3pt}
        	\item a fenti kívánalmaknak megfelel
        	        	
	    	\item a témában érintett szakterületek:
	    	\pause
	        \begin{itemize}
	            \item Épületgépészet
	            \item Szabályozástechnika
	            \item Jogszabályok, pénzügy és marketing 
	            %környezetgazdaságtan
	        \end{itemize}
        \end{itemize}
    \end{itemize}

\end{frame}

\begin{frame}{A munka célja}
\begin{itemize}
	\setlength{\itemsep}{6pt}
	\item Szabályozástechnikai tudás elményítése
	\item Kutatási eredmények megismerése
	\item Mindeközben törekvés a piacképességre is	
\end{itemize}
\end{frame}

\begin{frame}{Műszaki tartalom}

Piacvezérelt vagy tudásalapú terméket szeretnék?

\begin{itemize}
	\setlength{\itemsep}{7pt}
	\item tudományos: korszerű szabályozások, pl.
	\begin{itemize}
		\item optimális
		\item prediktív
		\item robosztus%szabályozástechnikai ismeretekre van szükség
	\end{itemize}
	\item piacivezérelt: igények alapján
	\begin{itemize}
		\item PI-szabályzós termosztát (önhangoló)
		\item intelligens otthonok (marketinggel fűszerezve)
	\end{itemize}
\end{itemize}
\end{frame}



\begin{frame}{Műszaki tartalom}

Piacvezérelt vagy tudásalapú terméket szeretnék?

\begin{itemize}
	\setlength{\itemsep}{7pt}
	\item Modellalapú szabályozás
	\begin{itemize}
		\item nagyobb komfort, alacsonyabb költségek
		\item innovatív, kutatják, publikálják az eredményeket
		\item komplex modellek, MIMO rendszerek kezelése
		\item optimalizációra visszavezethető beavatkozás\footnote{Az optimális beavatkozásnak sokféle kritériuma lehet.}
		%\item fontosabb a szabályozás minősége, mint hogy univerzális legyen
		%\item mérhető és nem mérhető zavarások figyelembe vétele
	\end{itemize}
	\item Kiindulás a piacon elérhető megoldásokból
	\begin{itemize}
		\item felkapott: intelligens otthon rendszerek
		\item multicégek termékei: Siemens, Bosch, Johnson Controls, Honeywell, Danfoss termosztátjai, okos rendszerei
		%\item fontosabb a szabályozás minősége, mint hogy univerzális legyen
		%\item mérhető és nem mérhető zavarások figyelembe vétele
	\end{itemize}
\end{itemize}
\end{frame}

\begin{frame}{Piacvezérelt termékfejlesztés}

Mire van igény a piacon?

\begin{itemize}
	\setlength{\itemsep}{7pt}
	\item Van egy problémakör:
	\begin{itemize}
		\item energiahatékonyság (törvényi megfelelőség)
		\item nagy kibocsátás
		\item magas költségek
		\item diszkomfort
	\end{itemize}
	\pause
	
	\item Megoldási lehetőség:
	\begin{itemize}
		\item egy korszerű fűtésszabályozás,\\
		ami teljesíti a követelményeket?
		%\item intelligens otthonok (marketinggel fűszerezve)
	\end{itemize}
\end{itemize}
\end{frame}



\begin{frame}{Tudásalapú termékfejlesztés}

Mit szeretnék csinálni?
\pause
\vspace{6pt}

\begin{itemize}
	\setlength{\itemsep}{12pt}
	\item Szabályozástechnika (analízis és tervezés):
	\begin{itemize}
		\item MIMO rendszerek paraméterbizonytalansággal
		\item mérhető vagy becsülhető zavarások
		\item prediktív szabályozás
	\end{itemize}
	\pause

	\item Fejlesztési lehetőség:
	\begin{itemize}
		\item egy korszerű fűtésszabályozás, \\
		amivel a fentiek vizsgálhatók, szemléltethetők?
		%\item intelligens otthonok (marketinggel fűszerezve)
	\end{itemize}
\end{itemize}
\end{frame}

\begin{frame}{A kiválasztott irány}

Szabályozástechnikai feladat:

\begin{itemize}
	\item helyiségenkénti hőmérsékletszabályozás,
	\item radiátoros és padlófűtéssel
\end{itemize}
\vspace{6pt}

Ehhez szükséges:


\begin{itemize}
	\item a szakasz paraméterezhető modellje
	\item egy modell-prediktív szabályozó
\end{itemize}


\end{frame}


\begin{frame}{MPC szabályozás}

Publikációk alapján a leggyakoribb korszerű szabályozó

\begin{itemize}
	\setlength{\itemsep}{12pt}
	\item Modellalapú működés - kép a Simulinkből
	\begin{itemize}
		\item épület
		\item fűtési rendszer
		\item prediktív szabályozás
	\end{itemize}
	\item Követelményei:
	\begin{itemize}
		\item radiátorszelep
		\item hőmérő
	\end{itemize}
\end{itemize}
\end{frame}



\begin{frame}{Szimuláció, modellalkotás}

A modell nagyon részletesen szerepel a dolgozatban, elvi újdonságot nem tartlalmaz (RC-hálózat) és a szabályozás rész érdekesebb, azzal foglalkoznék.

Tervezés szimulációval.

%\begin{itemize}
%\setlength{\itemsep}{10pt}
%\item Szabályozástechnika (analízis és tervezés):
%\begin{itemize}
%	\item MIMO rendszerek paraméterbizonytalansággal
%	\item mérhető vagy becsülhető zavarások
%	\item prediktív szabályozás
%\end{itemize}
%\item Fejlesztési lehetőség:
%\begin{itemize}
%	\item egy korszerű fűtésszabályozás?
%	%\item intelligens otthonok (marketinggel fűszerezve)
%\end{itemize}
%\end{itemize}


\end{frame}

\begin{frame}{Tervezés lépései}

Szimuláció:
\begin{itemize}
	\item valós rendszer modelljének paraméterezése
	\item modell identifikáció
	\item szabályozás tervezése, validálása
\end{itemize}
\vspace{6pt}

Valós rendszerre:
\begin{itemize}
	\item a tervezett szabályozó kipróbálása
	\item finomítás
\end{itemize}


\end{frame}

\end{document}
